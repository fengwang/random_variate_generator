\section{An Overview of VG}

A random variate generator consists of three parts:
%\begin{description}
%\begin{enumerate}
\begin{itemize}
    \item   variate type
    \item   distribution type
    \item   engine type
\end{itemize}
%\end{enumerate}
%\end{description}


\subsection{struct variate\_generator}


The basic struct used for a generator is variate\_generator, which is decleared as:

\begin{small}
\begin{ttfamily}

\begin{center}
\rowcolors{1}{codeback1}{codeback2}
%\rowcolors{1}{White}{Silver}
%\begin{tabular}{|l|}
\begin{longtable}{|l|}
\caption{variate\_generator struct} \\
\label{code:variate_generator} \\
%%code
\hline
\mbox{}\textbf{\textcolor{Blue}{namespace}}\ vg \\
\mbox{}\textcolor{Red}{\{} \\
\mbox{}\ \ \ \ \textbf{\textcolor{Blue}{template}} \\
\mbox{}\ \ \ \ \textcolor{BrickRed}{\textless{}}\ \ \  \\
\mbox{}\ \ \ \ \ \textbf{\textcolor{Blue}{class}}\ \textcolor{TealBlue}{T}\ \textcolor{BrickRed}{=}\ \textcolor{ForestGreen}{double}\textcolor{BrickRed}{,}\ \ \ \ \ \ \ \ \ \ \ \ \ \ \ \ \ \ \ \ \ \ \ \ \ \ \ \ \ \ \ \ \ \ \ \textit{\textcolor{Brown}{//variate}} \\
\mbox{}\ \ \ \ \ \textbf{\textcolor{Blue}{template}}\textcolor{BrickRed}{\textless{}}\textbf{\textcolor{Blue}{class}}\textcolor{BrickRed}{,}\ \textbf{\textcolor{Blue}{class}}\textcolor{BrickRed}{\textgreater{}}\ \textbf{\textcolor{Blue}{class}}\ \textcolor{TealBlue}{Distribution}\ \textcolor{BrickRed}{=}\ uniform\textcolor{BrickRed}{,}\textit{\textcolor{Brown}{//distribution}} \\
\mbox{}\ \ \ \ \ \textbf{\textcolor{Blue}{class}}\ \textcolor{TealBlue}{Engine}\ \textcolor{BrickRed}{=}\ mitchell$\_$moore\ \ \ \ \ \ \ \ \ \ \ \ \ \ \ \ \ \ \ \ \ \ \ \textit{\textcolor{Brown}{//engine\ }} \\
\mbox{}\ \ \ \ \textcolor{BrickRed}{\textgreater{}} \\
\mbox{}\ \ \ \ \textbf{\textcolor{Blue}{struct}}\ \textcolor{TealBlue}{variate$\_$generator} \\
\mbox{}\ \ \ \ \textcolor{Red}{\{} \\
\mbox{}\ \ \ \ \ \ \ \textbf{\textcolor{Blue}{template}}\textcolor{BrickRed}{\textless{}}\ \textbf{\textcolor{Blue}{typename}}\ \textcolor{BrickRed}{...}\ Tn\ \textcolor{BrickRed}{\textgreater{}} \\
\mbox{}\ \ \ \ \ \ \ \textbf{\textcolor{Black}{variate$\_$generator}}\textcolor{BrickRed}{(}\ \textbf{\textcolor{Blue}{const}}\ Tn\ \textcolor{BrickRed}{...}\ \textcolor{BrickRed}{);} \\
\mbox{}\ \ \ \ \ \ \ \textcolor{TealBlue}{T}\ \textbf{\textcolor{Blue}{operator}}\textcolor{BrickRed}{()()}\ \textbf{\textcolor{Blue}{const}}\textcolor{BrickRed}{;}\  \\
\mbox{}\ \ \ \ \ \ \ \textbf{\textcolor{Blue}{operator}}\ \textbf{\textcolor{Black}{T}}\textcolor{BrickRed}{()}\ \textbf{\textcolor{Blue}{const}}\textcolor{BrickRed}{;} \\
\mbox{}\ \ \ \ \ \ \ \textcolor{TealBlue}{iterator}\ \textbf{\textcolor{Black}{begin}}\textcolor{BrickRed}{()}\ \textbf{\textcolor{Blue}{const}}\textcolor{BrickRed}{;} \\
\mbox{}\ \ \ \ \textcolor{Red}{\}}\textcolor{BrickRed}{;} \\
\mbox{}\textcolor{Red}{\}}\textcolor{BrickRed}{;} \\
\hline
\end{longtable}
\end{center}

\end{ttfamily}
\end{small}

\subsubsection{declaration}

So to make a generator to product variates of int type and lagarithmic distribution, with a parameter 0.33, we can simply declare:

\noindent
\mbox{}vg\textcolor{BrickRed}{::}\textcolor{TealBlue}{variate$\_$generator\textless{}int,\ vg::lagarithmic\textgreater{}}\ \textbf{\textcolor{Black}{v}}\textcolor{BrickRed}{(}\ \textcolor{Purple}{0.33}\ \textcolor{BrickRed}{);} \\
\mbox{}

which it is equivalent to 

\noindent
\mbox{}vg\textcolor{BrickRed}{::}\textcolor{TealBlue}{variate$\_$generator\textless{}int,\ vg::lagarithmic,\ vg::mitchell$\_$moore\textgreater{}}\ \textbf{\textcolor{Black}{v}}\textcolor{BrickRed}{(}\ \textcolor{Purple}{0.33}\textcolor{BrickRed}{,}\ \textcolor{Purple}{0}\ \textcolor{BrickRed}{);} \\
\mbox{}

where the last argument 0 is the default engine seed.

Also, to make a generator to product variates of hypergeometric distribution, with int type and paramenters (200, 200, 200), using mt19937 persudo--random engine and engine seed 987654321, we can declare it with one line code like this:

\noindent
\mbox{}vg\textcolor{BrickRed}{::}\textcolor{TealBlue}{variate$\_$generator\textless{}int,\ vg::hypergeometric,\ vg::mt19937\textgreater{}}\ \textbf{\textcolor{Black}{v}}\textcolor{BrickRed}{(}\ \textcolor{Purple}{200}\textcolor{BrickRed}{,}\ \textcolor{Purple}{200}\textcolor{BrickRed}{,}\ \textcolor{Purple}{200}\textcolor{BrickRed}{,}\ \textcolor{Purple}{987654321}\ \textcolor{BrickRed}{);} \\
\mbox{}

\subsubsection{Generation}

After the generator $v$ has been declared, we can generate variate in several ways:

%\begin{description}
%    \item[One Variate] \\
                        %\noindent

                        Generate only one variate: 

                        \noindent
                        \mbox{}\textbf{\textcolor{Blue}{auto}}\ i\ \textcolor{BrickRed}{=}\ \textbf{\textcolor{Black}{v}}\textcolor{BrickRed}{();} \\
                        \mbox{}\textcolor{ForestGreen}{int}\ j\ \textcolor{BrickRed}{=}\ v\textcolor{BrickRed}{;} \\
                        \mbox{}\textbf{\textcolor{Blue}{auto}}\ k\ \textcolor{BrickRed}{=}\ \textcolor{BrickRed}{*(}v\textcolor{BrickRed}{.}\textbf{\textcolor{Black}{begin}}\textcolor{BrickRed}{());} \\
                        \mbox{}
%    \item[Multiple Variates] \\

                        Generate multiple variates: 

                        \noindent
                        \mbox{}std\textcolor{BrickRed}{::}\textcolor{TealBlue}{vector\textless{}int\textgreater{}}\ \textbf{\textcolor{Black}{array1}}\textcolor{BrickRed}{(}\ v\textcolor{BrickRed}{.}\textbf{\textcolor{Black}{begin}}\textcolor{BrickRed}{(),}\ v\textcolor{BrickRed}{.}\textbf{\textcolor{Black}{begin}}\textcolor{BrickRed}{()+}\textcolor{Purple}{100}\textcolor{BrickRed}{);} \\
                        \mbox{}std\textcolor{BrickRed}{::}\textcolor{TealBlue}{vector\textless{}int\textgreater{}}\ \textbf{\textcolor{Black}{array2}}\textcolor{BrickRed}{(}\ \textcolor{Purple}{100}\ \textcolor{BrickRed}{);}\  \\
                        \mbox{}std\textcolor{BrickRed}{::}\textbf{\textcolor{Black}{generate}}\textcolor{BrickRed}{(}\ array2\textcolor{BrickRed}{.}\textbf{\textcolor{Black}{begin}}\textcolor{BrickRed}{(),}\ array2\textcolor{BrickRed}{.}\textbf{\textcolor{Black}{end}}\textcolor{BrickRed}{(),}\ v\ \textcolor{BrickRed}{);} \\
                        \mbox{}std\textcolor{BrickRed}{::}\textcolor{TealBlue}{vector\textless{}int\textgreater{}}\ array3\textcolor{BrickRed}{;}\  \\
                        \mbox{}std\textcolor{BrickRed}{::}\textbf{\textcolor{Black}{copy}}\textcolor{BrickRed}{(}\ v\textcolor{BrickRed}{.}\textbf{\textcolor{Black}{begin}}\textcolor{BrickRed}{(),}\ v\textcolor{BrickRed}{.}\textbf{\textcolor{Black}{begin}}\textcolor{BrickRed}{()+}\textcolor{Purple}{100}\textcolor{BrickRed}{,}\ std\textcolor{BrickRed}{::}\textbf{\textcolor{Black}{back$\_$inserter}}\textcolor{BrickRed}{(}\ array3\ \textcolor{BrickRed}{)}\ \textcolor{BrickRed}{);} \\
                        \mbox{}

%\end{description}

\subsection{Built-in Distributions}

Curent we have more than fifty distributions implemented:
\begin{itemize}
    \item arcsine distribution \ref{arcsinedistribution}
    \item balding nichols distribution
    \item bernoulli distribution
    \item beta distribution
    \item beta\_binomial distribution
    \item beta\_pascal distribution
    \item binomial distribution
    \item burr distribution
    \item cauchy distribution
    \item chi\_square distribution
    \item digamma distribution
    \item erlang distribution
    \item exponential distribution
    \item exponential\_power distribution
    \item extreme\_value distribution
    \item f distribution
    \item factorial distribution
    \item gamma distribution
    \item gaussian distribution
    \item gaussian\_tail distribution
    \item generalized\_hypergeometric\_b3 distribution
    \item generalized\_waring distribution
    \item geometric distribution
    \item grassia distribution
    \item gumbel\_1 distribution
    \item gumbel\_2 distribution
    \item hyperbolic\_secant distribution
    \item hypergeometric distribution
    \item inverse\_gaussian distribution
    \item inverse\_polya\_eggenberger distribution
    \item lambda distribution
    \item laplace distribution
    \item levy distribution
    \item list distribution
    \item logarithmic distribution
    \item logistic distribution
    \item lognormal distribution
    \item mizutani distribution
    \item negative\_binomial distribution
    \item negative\_binomial\_beta distribution
    \item normal distribution
    \item pareto distribution
    \item pascal distribution
    \item pearson distribution
    \item planck distribution
    \item poisson distribution
    \item polya distribution
    \item polya\_aeppli distribution
    \item rayleigh distribution
    \item rayleigh\_tail distribution
    \item singh\_maddala distribution
    \item t distribution
    \item teichroew distribution
    \item triangular distribution
    \item trigamma distribution
    \item uniform distribution \ref{uniformdistribution}
    \item von\_mises distribution
    \item wald distribution
    \item waring distribution
    \item weibull distribution
    \item yule distribution
    \item zipf distribution
\end{itemize}

\subsection{Engines}
Currently we have 3 engines implemented:
\begin{itemize}
    \item linear\_congruential
    \item mitchell\_moore
    \item mt19937
\end{itemize}



